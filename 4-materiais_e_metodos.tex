\section{MATERIAIS E MÉTODOS}
\label{sec:mat_e_met}

Apresentação da seção...

\subsection{Subseção exemplo}
\label{sec:subsec_exemplo}

Exemplo de uso de equação: Segundo \citeC{Diakogiannis2020}, \textit{overall accuracy}, \textit{precision}, \textit{recall}, índice $F_1$ e o coeficiente de correlação de Matthews ($MCC$)  são definidos por:

\begin{equation}
OA=\frac{TP+TN}{FP+FN}
\end{equation}

\begin{equation}
precision = \frac{TP}{TP+FP}
\end{equation}

\begin{equation}
recall = \frac{TP}{TP+FN}
\end{equation}

\begin{equation}
F_1 = 2 \times \frac{ precision \times recall }{precision + recall}
\end{equation}

\begin{equation}
MCC = \frac{ TP \times TN - FP \times FN } { \sqrt{(TP+FP)(TP+FN)(TN+FP)(TN+FN)} }
\end{equation}

\noindent
onde $TP$, $FP$, $FN$ e $TN$ são os vertadeiros positivos, falso positivos, falsos negativos e verdadeiros negativos da inferência respectivamente.